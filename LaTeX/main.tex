% filepath: c:\Users\Le R\Desktop\Code\Python\Coalescent\LaTeX\main.tex
% ...existing code...
\documentclass[12pt,a4paper]{report}

\usepackage[utf8]{inputenc}
\usepackage[T1]{fontenc}
\usepackage[french]{babel}
\usepackage{lmodern}
\usepackage{microtype}
\usepackage{geometry}
\usepackage{graphicx}
\usepackage{amsmath,amssymb}
\usepackage{hyperref}
\usepackage{caption}

\geometry{margin=2.5cm}

\title{Titre du rapport}
\author{Nom de l'auteur}
\date{\today}

\begin{document}

\maketitle

\section{Introduction}
DRVJKEL

\appendix
\chapter{Annexes}
% ...matériel complémentaire...

\chapter{Déployer ce dossier sur GitHub}
Voici une procédure minimale pour transformer ce dossier local en dépôt Git suivi sur GitHub :

\begin{enumerate}
  \item Ouvrir un terminal dans le dossier du projet.
  \item Initialiser git :
  \begin{verbatim}
  git init
  git add .
  git commit -m "Initial commit"
  \end{verbatim}
  \item Créer le dépôt distant sur GitHub (via le site web) ou avec l'outil gh :
  \begin{verbatim}
  gh repo create NOM_UTILISATEUR/NOM_REPO --public --source=. --remote=origin
  \end{verbatim}
  (ou créez manuellement sur github.com puis copiez l'URL)
  \item Lier et pousser :
  \begin{verbatim}
  git branch -M main
  git remote add origin git@github.com:UTILISATEUR/NOM_REPO.git
  git push -u origin main
  \end{verbatim}
  \item Exemple de .gitignore minimal pour LaTeX/Python :
  \begin{verbatim}
  *.aux
  *.log
  *.out
  *.toc
  __pycache__/
  *.pyc
  .vscode/
  .DS_Store
  \end{verbatim}
\end{enumerate}

\end{document}