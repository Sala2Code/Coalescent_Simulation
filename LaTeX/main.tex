\documentclass[12pt,a4paper]{article}

\usepackage[T1]{fontenc}
\usepackage[utf8]{inputenc}
\usepackage[french]{babel}
\usepackage{lmodern}
\usepackage{microtype}
\usepackage{geometry}
\usepackage{graphicx}
\usepackage{amsmath,amssymb,amsthm}
\usepackage{stmaryrd}
\usepackage{caption}
\usepackage{natbib}
\usepackage{hyperref} % garder en dernier
\usepackage{xcolor}

\geometry{margin=1cm}

\newtheorem{theorem}{Théorème}
\newtheorem{definition}{Définition}
\newtheorem{proposition}{Proposition}

\title{Titre du rapport}
\author{Nom de l'auteur}
\date{\today}

\begin{document}

\maketitle

\section{Introduction à la théorie de la coalescence.}

\subsection{Kingman coalescent est la limite de Wright-Fisher.}
Nous disposons d'une population haploïde \footnote{Chaque individu porte une seule copie
du génome, modélisée par un unique parent.}
de taille $N$. On commence à la génération $m=0$ et nous souhaitons remonter le temps pour
étudier la généalogie de $n < N$ individus, généralement on considère $n \ll N$.
On appelle la lignée de $i \in [n]:=\llbracket 1,n \rrbracket$ la suite
des ancêtres de $i$.
Lorsque deux, ou plus, lignées choisissent le même parent à un instant donné, on dit qu'elles coalescent.

Formellement, on définit pour la génération $m \in \mathbb{N}$,
la relation d'équivalence $\sim_m$ pour tout $i,j \in [n]$,
$$
i \sim_m j \text{ ssi} \text{ les lignées de } i \text{ et } j \text{ ont le même ancêtre à la génération m} 
$$

La classe de $i$ à la génération $m$, constituant un bloc, est,
$$
C_m^{(N)}(i):=\{ j\in[n], i\sim_m j \}
$$

Puis, la famille des classes distinctes forme une partition de [n]
\[
\Pi_m^{(N)}:=\{C_m^{(N)}(i), i\in[n]\} 
\qquad
K_m:=|\Pi_m^{(N)}|
\]

% En remontant le temps, les partitions ne peuvent que se grossir
% \[
% \Pi_0^{(N)}\ \preceq\ \Pi_1^{(N)}\ \preceq\ \Pi_2^{(N)}\ \preceq\ \cdots
% \]
% (où $\preceq$ est l’ordre de raffinement).

Dans le modèle historique de Wright-Fisher, à chaque génération,
chaque individu choisit un parent au hasard dans la génération précédente, de manière
indépendante et uniforme \cite{Fisher1930,Wright1931}. On a ainsi que
$(\Pi_m^{(N)})_{m\ge 0}$ est une chaîne de Markov homogène à temps discret sur les
partitions de $[n]$.

Notons $(\cdot)_k = \prod_{i=\cdot-k+1}^{\cdot} i$.
On a que la probabilité que $k\geq 2$ lignées de $b$ blocs coalescent en une génération est donnée par,  
$$
\binom{b}{k}\frac{(N)_{b-k+1}}{N^{b}} = \begin{cases}
  \binom{b}{2}\frac{1}{N} + \mathcal{O} (\frac{1}{N^2}) & \text{ si } k=2 \\
  \mathcal{O} (\frac{1}{N^2}) & \text{ si } b \geq k \geq 3 \\
  0 & \text{ sinon }

\end{cases}
$$

On remarque donc que l'évènement d'une coalescence est rare lorsque $N$ est grand.
On considère ainsi le processus de Markov $(\Pi^{(N)}(t))_{t\geq 0}$ défini par,
$$
\Pi^{(N)} : t \in \mathbb{R}^+ \longmapsto \Pi^{(N)}_{\lfloor tN \rfloor}
$$

Notons le semi-groupe associé
\[
P_t^N : (x,y) \mapsto \mathbb{P}( \Pi^{(N)}(t) = y | \Pi^{(N)}(0) = x)
\]

Considérons $x$ une partition de $[n]$ avec $k$ blocs, 
$y$ une partition de $[n]$, avec $k-1$ blocs, obtenue en fusionnant deux blocs de $x$, 
et $y^+$ une partition de $[n]$
en fusionnant plus de deux blocs de $x$.
Puisque $t \mapsto P_t^N$ est en escalier, le générateur infinitésimal est donné par,
\[
q^{(N)} : (x,z) \longmapsto N(P_{1/N}^N(x,z) - \delta_{x,z}) =
\left\{
\begin{array}{rcll}
N\biggl( \frac{1}{N} + \mathcal{O}(\tfrac{1}{N^2})- 0 \biggr) & = & 1 + \mathcal{O}(\tfrac{1}{N}) & \text{ si } z=y \\
N\biggl( \mathcal{O}(\tfrac{1}{N^2}) - 0 \biggr) & = & \mathcal{O}(\tfrac{1}{N}) & \text{ si } z=y^+ \\
N\biggl( 1 - \binom{k}{2}\tfrac{1}{N} + \mathcal{O}(\tfrac{1}{N^2}) - 1 \biggr) & = & - \binom{k}{2} + \mathcal{O}(\tfrac{1}{N}) & \text{ si } z=x
\end{array}
\right.
% \begin{cases}
%   1 + \mathcal{O}(\frac{1}{N}) & \text{ si } z=y \\
%   \mathcal{O}(\frac{1}{N}) & \text{ si } z=y^+ \\
%   - \binom{k}{2} + \mathcal{O}(\frac{1}{N})  & \text{ si } z=x  \\
% \end{cases}
\]

En passant à la limite sur $N$ vers l'infini,
nous obtenons le coalescent de Kingman \cite{Kingman1982}. Son générateur est donné par, 
\[
q : (x,z) \longmapsto \lim_{N \to \infty} q^{(N)}(x,z) =
\begin{cases}
  1 & \text{ si } z=y \\
  - \binom{k}{2} & \text{ si } z=x \\
  0 & \text{ sinon }
\end{cases}
\]

Ce modèle considère uniquement le fait que deux lignées peuvent coalescer à la fois.

\subsection{$\Lambda$-Coalescent.}

Intéressons nous à un cas plus général
où un certain nombre de lignées peuvent coalescer simultanément.
Nous restons dans l'esprit d'un modèle limite où $N$ tend vers l'infini
afin de considérer une proportion $x \in [0,1]$ de la population descendant d'un même individu. 
La description du générateur infinitésimal de ce processus
de Markov repose sur le théorème suivant,

\begin{theorem}[Pitman-Sagitov {\cite{Pitman1999,Sagitov1999}}]
Il existe un processus de Markov, appelé $\Lambda$-coalescent,
échangeable à collisions multiples simples si et seulement s'il
existe une mesure finie $\Lambda$ sur $[0,1]$ telle que, lorsqu'on a $b$ blocs,
pour tout $2\le k\le b$ le taux auquel chaque $k$-uplet fixé de blocs
fusionne vaut, 
\[
\lambda_{b,k}=\int_0^1 x^{k-2}(1-x)^{b-k}\,\Lambda(dx)
\]
\end{theorem}

Sous des hypothèses raisonnables telles 
que l'absence de mémoire (processus de Markov),
la permutation invariante des blocs (échangeabilité) et 
qu'à chaque instant il ne peut pas y avoir plusieurs coalescences 
n'ayant pas le même ancêtre (collisions multiples simples), 
ce résultat montre que la dynamique 
d'un processus de coalescence multiple est entièrement caractérisée
par une mesure finie $\Lambda$ sur $[0,1]$.

Dans ce rapport nous simulons différents $\Lambda$-coalescents
pour différentes mesures $\Lambda$
et \textcolor{red}{TODO : dire ce qu'on veut faire réellement que juste faire des plots}.

% On remarque alors qu'afin de satisfaire de raisonnable hypothèses le taux de coalescence
% doit être de la forme suivante, pour $\Lambda$ une mesure finie sur $[0,1]$,
% $k$ le nombre de blocs actuels et $l$ le nombre de blocs coalescents,
% alors le taux de coalescence est,
% $$
% \lambda_{k,l} = \int_0^1 x^{l-2}(1-x)^{k-l} \Lambda(dx)
% $$

\section{\textcolor{red}{??}}
\textcolor{red}{Je sais pas si on fait une "liste" de mesure qu'on regarde,
je pensais plutôt renrer dans le vif du sujet avec la problémtique définie
et au fur et à mesure on montre les différents plot avec les différentes mesures 
testées.
}

\textcolor{red}{Parler de Lambda = $\delta_0$ : Kingman (dire que la theorie "généralise") + Parler de beta ?}

\textcolor{red}{TODO :
\begin{itemize}
  \item Image vulgarisatrice de ce qu'on fait (j'ai une idée donc à voir si c'est utile)
  \item Détails des calculs d'Axcel pour différentes mesure : nouvelle section + subsections
  \item Enchainer avec une succession de beaux plots
  \item Lancer une problématique
  (à en discuter voir où on veut aller :
  trouver un paramètre par simulation
  (temps d'un unique ancêtre, pb de coalescence multiple, \ldots),
  estimer la distance entre deux modèles (KS, Wasserstein, \ldots),
  usage détourné de la coalescence pour faire quelque chose
  d'atypique)
\end{itemize}
}

\bibliographystyle{unsrtnat} % ou abbrvnat, unsrtnat, etc.
\bibliography{ref}           % ref.bib dans le projet
\end{document}