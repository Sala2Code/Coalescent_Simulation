\documentclass[12pt,a4paper]{article}

\usepackage[T1]{fontenc}
\usepackage[utf8]{inputenc}
\usepackage[french]{babel}
\usepackage{lmodern}
\usepackage{microtype}
\usepackage{geometry}
\usepackage{graphicx}
\usepackage{amsmath,amssymb,amsthm}
\usepackage{stmaryrd}
\usepackage{caption}
\usepackage{natbib}
\usepackage{hyperref} % garder en dernier
\usepackage{xcolor}

\geometry{margin=1cm}

\newtheorem{theorem}{Théorème}
\newtheorem{definition}{Définition}
\newtheorem{proposition}{Proposition}

\title{Titre du rapport}
\author{Nom de l'auteur}
\date{\today}

\begin{document}

\maketitle

\subsection{$\Lambda$-Coalescent.}

\textcolor{red}{Vu que la première sous section est supprimée,
je développerai légrement cette sous-section.  
Voir la TODO list (TODO.md) afin de savoir faire.}

Intéressons nous à un cas plus général
où un certain nombre de lignées peuvent coalescer simultanément.
Nous restons dans l'esprit d'un modèle limite où $N$ tend vers l'infini
afin de considérer une proportion $x \in [0,1]$ de la population descendant d'un même individu. 
La description du générateur infinitésimal de ce processus
de Markov repose sur le théorème suivant,

\begin{theorem}[Pitman-Sagitov {\cite{Pitman1999,Sagitov1999}}]
Il existe un processus de Markov, appelé $\Lambda$-coalescent,
échangeable à collisions multiples simples si et seulement s'il
existe une mesure finie $\Lambda$ sur $[0,1]$ telle que, lorsqu'on a $b$ blocs,
pour tout $2\le k\le b$ le taux auquel chaque $k$-uplet fixé de blocs
fusionne vaut, 
\[
\lambda_{b,k}=\int_0^1 x^{k-2}(1-x)^{b-k}\,\Lambda(dx)
\]
\end{theorem}

Sous des hypothèses raisonnables telles 
que l'absence de mémoire (processus de Markov),
la permutation invariante des blocs (échangeabilité) et 
qu'à chaque instant il ne peut pas y avoir plusieurs coalescences 
n'ayant pas le même ancêtre (collisions multiples simples), 
ce résultat montre que la dynamique 
d'un processus de coalescence multiple est entièrement caractérisée
par une mesure finie $\Lambda$ sur $[0,1]$.

Dans ce rapport nous simulons différents $\Lambda$-coalescents
pour différentes mesures $\Lambda$
et \textcolor{red}{TODO : dire ce qu'on veut faire réellement que juste faire des plots}.

% On remarque alors qu'afin de satisfaire de raisonnable hypothèses le taux de coalescence
% doit être de la forme suivante, pour $\Lambda$ une mesure finie sur $[0,1]$,
% $k$ le nombre de blocs actuels et $l$ le nombre de blocs coalescents,
% alors le taux de coalescence est,
% $$
% \lambda_{k,l} = \int_0^1 x^{l-2}(1-x)^{k-l} \Lambda(dx)
% $$

% \section{\textcolor{red}{??}}
% \textcolor{red}{Je sais pas si on fait une "liste" de mesure qu'on regarde,
% je pensais plutôt renrer dans le vif du sujet avec la problémtique définie
% et au fur et à mesure on montre les différents plot avec les différentes mesures 
% testées.
% }

% \textcolor{red}{Parler de Lambda = $\delta_0$ : Kingman (dire que la theorie "généralise") + Parler de beta ?}

% \textcolor{red}{TODO :
% \begin{itemize}
%   \item Image vulgarisatrice de ce qu'on fait (j'ai une idée donc à voir si c'est utile)
%   \item Détails des calculs d'Axcel pour différentes mesure : nouvelle section + subsections
%   \item Enchainer avec une succession de beaux plots
%   \item Lancer une problématique
%   (à en discuter voir où on veut aller :
%   trouver un paramètre par simulation
%   (temps d'un unique ancêtre, pb de coalescence multiple, \ldots),
%   estimer la distance entre deux modèles (KS, Wasserstein, \ldots),
%   usage détourné de la coalescence pour faire quelque chose
%   d'atypique)
% \end{itemize}
% }

\bibliographystyle{unsrtnat} % ou abbrvnat, unsrtnat, etc.
\bibliography{ref}           % ref.bib dans le projet
\end{document}