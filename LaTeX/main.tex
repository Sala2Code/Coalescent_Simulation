\documentclass[12pt,a4paper]{report}

\usepackage[utf8]{inputenc}
\usepackage[T1]{fontenc}
\usepackage[french]{babel}
\usepackage{lmodern}
\usepackage{microtype}
\usepackage{geometry}
\usepackage{graphicx}
\usepackage{amsmath,amssymb}
\usepackage{hyperref}
\usepackage{caption}
\usepackage{color}
\usepackage{stmaryrd}

\geometry{margin=1cm}

\title{Titre du rapport}
\author{Nom de l'auteur}
\date{\today}

\begin{document}

\maketitle

\section{Introduction à la théorie de la coalescence.}

Nous disposons d'une population haploïde \footnote{Chaque individu porte une seule copie
du génome, modélisée par un unique parent.}
de taille $N$. On commence à la génération $m=0$ et nous souhaitons remonter le temps pour
étudier la généalogie de $n < N$ individus, généralement on considère $n \ll N$.
On appelle la lignée de $i \in [n]:=\llbracket 1,n \rrbracket$ la suite
des ancêtres de $i$.
Lorsque deux, ou plus, lignées choisissent le même parent à un instant donné, on dit qu'elles coalescent.

Formellement, on définit pour la génération $m \in \mathbb{N}$,
la relation d'équivalence $\sim_m$ pour tout $i,j \in [n]$,
$$
i \sim_m j \text{ ssi} \text{ les lignées de } i \text{ et } j \text{ ont le même ancêtre à la génération m} 
$$

La classe de $i$ à la génération $m$, constituant un bloc, est,
$$
C_m^{(N)}(i):=\{ j\in[n], i\sim_m j \}
$$

Puis, la famille des classes distinctes forme une partition de [n]
\[
\Pi_m^{(N)}:=\{C_m^{(N)}(i), i\in[n]\} 
\qquad
K_m:=|\Pi_m^{(N)}|
\]

% En remontant le temps, les partitions ne peuvent que se grossir
% \[
% \Pi_0^{(N)}\ \preceq\ \Pi_1^{(N)}\ \preceq\ \Pi_2^{(N)}\ \preceq\ \cdots
% \]
% (où $\preceq$ est l’ordre de raffinement).

Dans le modèle historique de Wright-Fisher, à chaque génération,
chaque individu choisit un parent au hasard dans la génération précédente, de manière
indépendante et uniforme. On a ainsi que
$(\Pi_m^{(N)})_{m\ge 0}$ est une chaîne de Markov homogène à temps discret sur les
partitions de $[n]$.

Notons $(\cdot)_k = \prod_{i=\cdot-k+1}^{\cdot} i$.
On a que la probabilité que $l\geq 2$ lignées de $k$ blocs coalescent en une génération est donnée par,  
$$
\binom{k}{l}\frac{(N)_{k-l+1}}{N^{k}} = \begin{cases}
  \binom{k}{2}\frac{1}{N} + \mathcal{O} (\frac{1}{N^2}) & \text{ si } l=2 \\
  \mathcal{O} (\frac{1}{N^2}) & \text{ si } k \geq l \geq 3 \\
  0 & \text{ sinon }

\end{cases}
$$

On remarque donc que l'évènement d'une coalescence est rare lorsque $N$ est grand.
On considère ainsi le processus de Markov $(\Pi^{(N)}(t))_{t\geq 0}$ défini par,
$$
\Pi^{(N)} : t \in \mathbb{R}^+ \longmapsto \Pi^{(N)}_{\lfloor tN \rfloor}
$$

Notons le semi-groupe associé
\[
P_t^N : (x,y) \mapsto \mathbb{P}( \Pi^{(N)}(t) = y | \Pi^{(N)}(0) = x)
\]

Considérons $x$ une partition de $[n]$ avec $k$ blocs, 
$y$ une partition de $[n]$, avec $k-1$ blocs, obtenue en fusionnant deux blocs de $x$, 
et $y^+$ une partition de $[n]$
en fusionnant plus de deux blocs de $x$.
Puisque $t \mapsto P_t^N$ est en escalier, le générateur infinitésimal est donné par,
\[
q^{(N)} : (x,z) \longmapsto N(P_{1/N}^N(x,z) - \delta_{x,z}) =
\left\{
\begin{array}{rcll}
N\biggl( \frac{1}{N} + \mathcal{O}(\tfrac{1}{N^2})- 0 \biggr) & = & 1 + \mathcal{O}(\tfrac{1}{N}) & \text{ si } z=y \\
N\biggl( \mathcal{O}(\tfrac{1}{N^2}) - 0 \biggr) & = & \mathcal{O}(\tfrac{1}{N}) & \text{ si } z=y^+ \\
N\biggl( 1 - \binom{k}{2}\tfrac{1}{N} + \mathcal{O}(\tfrac{1}{N^2}) - 1 \biggr) & = & - \binom{k}{2} + \mathcal{O}(\tfrac{1}{N}) & \text{ si } z=x
\end{array}
\right.
% \begin{cases}
%   1 + \mathcal{O}(\frac{1}{N}) & \text{ si } z=y \\
%   \mathcal{O}(\frac{1}{N}) & \text{ si } z=y^+ \\
%   - \binom{k}{2} + \mathcal{O}(\frac{1}{N})  & \text{ si } z=x  \\
% \end{cases}
\]

En passant à la limite sur $N$ vers l'infini,
nous obtenons le coalescent de Kingman. Son générateur est donné par, 
\[
q : (x,z) \longmapsto \lim_{N \to \infty} q^{(N)}(x,z) =
\begin{cases}
  1 & \text{ si } z=y \\
  - \binom{k}{2} & \text{ si } z=x \\
  0 & \text{ sinon }
\end{cases}
\]


Ce modèle considère uniquement le fait que deux lignées peuvent coalescer à la fois.
Nous pourrions envisager des généralisations où plusieurs lignées
coalescent simultanément. [INTRODUIRE NATURELLEMENT LE LAMBA-COALESCENT]


\textcolor{red}{TODO : 
\begin{itemize}
  \item Image vulgarisatrice de ce qu'on fait (j'ai une idée donc à voir si c'est utile)
  \item  Détails des calculs d'Axcel pour différentes mesure : nouvelle section + subsections
\item Enchainer avec une succession de beau plot 
\item lancer une problématique
(à en discuter voir où on veut aller :
trouver un parametre par simulation
(temps d'un unique ancetre, pb de coalescence multiple, ... ),
estimer la distance entre deux modèles (KS, Wasserstein, ...),
usage détourner de la coalescence pour faire quelquechose
d'atypiques ) 
\end{itemize}
}
\end{document}